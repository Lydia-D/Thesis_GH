%!TEX root = ../Thesis.tex

\section{Moving Forward} % or is it conclusion?
GPS is currently going through a major upgrade, with the launch of new satellites with alternate structuring of signals that will provide greater accuracy for civil use in the future. The constellation Galileo that is currently being developed and launched, is expected to be finished by XX. Both the new GPS and Galileo are designed to be able to operate in tandem to provide an *unprecedented* accuracy with a total of 57 operational joint satellites, while still having the individual capability of global coverage. Some technology today uses both GPS and GLONASS constellations but requires duel receivers as the signals are encoded differently. GPS uses code division multiple access (CMDA) whereas GLONASS uses frequency division multiple access (FMDA). Research is also being conducted on the sources of error in the radio signals. The errors are to be modelled and the appropriate adjustments stored in the encoding of the signal. Through the international cooperation efforts and research, localisation and tracking using baseline GNSS will continue to improve, regardless of alternate data processing algorithms. For now, improved signal accuracy will obviously improve the performance of the different data processing algorithms. However, there will come a time where the extra computational and hardware complexity will not be worth the small gain in accuracy over the default performance of the low-cost civilian GNSS receiver.
