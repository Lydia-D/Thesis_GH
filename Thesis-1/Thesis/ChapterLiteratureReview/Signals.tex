%!TEX root = ../Thesis.tex

\section{GPS Satellite Signals}
The are two frequencies that GPS uses; L1 frequency at 1575.42 MHz and L2 at 1227.60 MHz \cite{postprocessing multi}.

There are two sets of signals that are sent from every satellite, the pseudorandom binary sequence (PRN code) and the navigation message.

\subsection{Course Acquisition Code}
The course acquisition code (C/A) 

Low cost GPS receivers only have access to 

The PRN code is intentionally complex to stop jamming 

L1 frequency of 1575.42 Mhz  - code phase
Course Acquisition (C/A) code is transmitted on the L1 frequency as 1.023 MHz signal using a bi-phase shift keying modulation technique.
Navigation message sent at 50 bits per second  


\subsection{Navigation Message}
The navigation message is a low 

\url{https://ocw.mit.edu/courses/earth-atmospheric-and-planetary-sciences/12-540-principles-of-the-global-positioning-system-spring-2012/lecture-notes/MIT12_540S12_lec7.pdf}

- civilian GNSS using duel frequency, send CDMA, how decryption works\\
- what is psudorange?\\
- what is carrier phase\\
- clock bias\\

\subsection{Raw Data}
There is a lot of data contained in the total signal, but the following are what is important to this thesis:
\begin{itemize}
	\item \textbf{Time received}: The time in the receivers frame that the sample reading was taken.
	\item \textbf{Pseudorange}: The range calculated by the receiver to the satellite. Depending upon the type of receiver, this measurement may have already been adjusted for some errors that were encoded in the navigation message.
	\item \textbf{Carrier Phase}: The phase of the carrier signal at the receivers point in time.
	\item \textbf{Doppler Shift}: The instantaneous Doppler frequency of the signal. 
	\item \textbf{Satellite Epoch}: The time the signal was sent from the satellite, decoded from the navigation message.
	\item \textbf{Ephemeris Data}: The orbital parameters necessary to calculate the position of the satellite.
\end{itemize}


%The first row shows a C/A code with 1,023 chips; the total length is 1 ms. The second row shows a navigation data bit that has a data rate of 50 Hz; thus, a data bit is 20 ms long and contains 20 C/A codes. Thirty data bits make a word that is 600 ms long as shown in the third row. Ten words make a subframe that is 6 seconds long as shown in row four. The fifth row shows a page that is 30 seconds long and contains 5 subframes. Twenty-five pages make a complete data set that is 12.5 minutes long as shown in the sixth row. The 25 pages of data can be referred to as a superframe http://read.pudn.com/downloads85/ebook/326017/Fundamentals%20of%20Global%20Positioning%20System%20Receivers/booktext05.pdf (pg77)

% resources about the signals:
% http://geoconnect.com.au/gps-signals-l1-l2-l5/
% http://www.trimble.com/gps_tutorial/dgps-advanced4.aspx
% https://www.e-education.psu.edu/natureofgeoinfo/c5_p14.html
% http://what-when-how.com/gps/gps-details/
% https://www.e-education.psu.edu/geog862/node/1742
% https://ocw.mit.edu/courses/earth-atmospheric-and-planetary-sciences/12-540-principles-of-the-global-positioning-system-spring-2012/lecture-notes/MIT12_540S12_lec7.pdf
% http://read.pudn.com/downloads85/ebook/326017/Fundamentals%20of%20Global%20Positioning%20System%20Receivers/booktext05.pdf
% http://www.navipedia.net/index.php/GPS_Navigation_Message