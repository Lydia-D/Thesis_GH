%!TEX root = ../Thesis.tex

\def\chapdir{./ChapterIntro}

\chapter{Introduction} \label{ch:intro}

%% Stefan Williams::
%Introduction
%1.1 Background and Motivation (General fluff then location and mapping)
%1.2 Problem Summary
%1.3 Principle Contributions
%1.4 Outline

%%=======================================================%%

\section{Background and Motivation}

Clearly identifies the problem for investigation and relevant context
Clearly sets out the content and aims of the project
Localisation, outdoors GNSS
- accuracy 

Localisation is an integral part of the modern world.

- GNSS - talk about different constellations then pick GPS as the topic as it is the most avalible atm but galelaio and gps will soon combine








\section{Problem Summary}
- get high drift using dead reckoning in robotics data
- for outdoor solutions, gps is the easiest solution as it is already setup
- however due to the high error in accuracy, it doesn't really solve the drift problem small scale, other data is required such as laser beacons, it is only used as global drift correction
- embedded systems often have minimal computational space/time/processing available 
- current algorithms that have minimal setup have high computation costs
- simple differential algorithms have high setup that require access to a known node, or access to realtime error information from the internet

%- has low update rates compared to accelerometer data

what is the problem to be solved?

- find a way to have accurate (how accurate?) localisation data with minimal calibration or setup using low cost receivers and low computational requirements  
- a system that has requirements for high accuracy typically needs it for between parts of the system in which case there will be a receiver on each part, or be accurate to a specific location in th field, not to the global reference frame. Therefore differencing systems

\section{Principle Contributions}
The following list is an overview of what I have contributed to the field. 
\begin{itemize}
\item I carried out the literature survey in order to identify what areas of GNSS can be built on.
\item I made the conceptual breakdown of the planar algorithm (PA) and the construction of the residuals: It is then solved using the generic solution for an overdetermined, non-homogeneous, linear system using least squares.
\item I wrote a simulation in Matlab: The simulation calculates the pseudorange from all visible satellites to all receivers and adds randomised error in a controlled way to mimic different types of error. It implements the epoch alignment and planar algorithm with a comparison to 
\item I used existing subfunctions as a part of my simulation that I had previously written: The subfunctions were all written as a part of Assignment 2 of the unit of study AERO4701 Space Engineering 3 in Semester 1 2016. The code that was adapted were; coordinate frame transforms, creation of plots in polar and cartesian frames, GPS constellation data, non-linear least squares solution of absolute position using pseudoranges.
\item I carried out the analysis of the planar algorithm using the simulation as previously mentioned. The conclusions are my own.
\end{itemize}


\section{Outline}
The rest of the thesis is organised as follows. Chapter \ref{ch:litreview} is an overview of how GPS works including the operation components and signal structure. The sources of error will be introduced and relevant work in the literature of how the errors can be minimised. Chapter \ref{ch:perception} is a discussion of the assumptions and the detailed mathematical methodology of the planar algorithm. Chapter \ref{ch:experiments} is a full analysis of the planar algorithm and comparison to 




need instantanous relative positioning with minimal calibration or setup of external hardware

Instantaneous Relative displacement/position between GNSS receivers.\\
Simulation case studies are presented to validate the mathematical models.\\

The algorithm presented is designed to replace current methods.

With a local reference point with a well known global location, the relative position of the other receivers 