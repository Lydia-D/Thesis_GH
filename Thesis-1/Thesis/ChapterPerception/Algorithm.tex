%!TEX root = ../Thesis.tex
\section{Planar Intersection Algorithm}

- assume satellite is at infinity for comparing difference in pseudorange for a particular reference satellite.\\
- use all satellites as reference satellite - no single point of failure, also not all satellites might be in view for all receivers\\
- get the normal vector between all receivers and each sat. \\
- Calculate the average normal vector.\\
- get the difference in pseudorange between all receivers along each normal vector \\
- create a plane with the normal vector with that distance\\
- solve via optimization (least squares) \\
- use clock adjustment from abs gps? or have as another optimisation variable\\
- antenna problems? misalignment?\\
- share clock bias's between solving for different reference sets? - do it one by one or all together?\\
- need to align the time of signal sent to the receivers before calculating average normal vector\\

- have weighted planes based on ? have weighted area on the planes?\\
- if one plane intercepts far away from the others then ignore it (multipath). hyperdimensional surface to minimise




% conceptual
how to send data between receivers? do it offline on a different platform?


https://www.e-education.psu.edu/geog862/node/1759 - errors in pseduorange

http://www.insidegnss.com/node/2898 - how to get pseudorange from raw data




\subsection{Algorithm}

\subsubsection{Pre-Processing}
\paragraph{Select reference receiver $\alpha$}
The receiver $\alpha$ is used as the reference location and common time in the NED frame. 
\paragraph{Collect data of one timestep from all receivers}
The raw data as well as the estimated absolute location and clock bias (what frame of reference is this?) from non-linear least squares optimisation is collected from all GNSS receivers.
\paragraph{Align to reference Epoch time}\label{timetransform}


\subsubsection{Distance Optimisation}
By optimising the distance between each pair of receivers, the error in the whole system is minimised. 
This means that the position receivers are not only relative to the reference receiver $\alpha$ but between all receivers just with the reference frame origin located at $\alpha$. It is because of this step a receiver does not need to have all the same satellites in view as all other receivers, including the designated $\alpha$.


\paragraph{Average normal Vector}
Find the average normal vector pointing to each satellite $\hat{\eta_s}$ from the receivers. The normal vector is calculated by using the position all of the satellites in view at the common time $t_{\alpha}$ as previously transformed in \ref{timetransform} and the estimated absolute position of all receivers. The average for each satellite is calculated by taking the mean across all receivers.

\paragraph{Difference in Pseudorange}
The differences in pseudorange are calculated $\Delta\rho^s_{\omega_i\omega_j}$ where s is the satellite, $\omega_i$ and $\omega_j$ are receivers $(for i<j, i\neq j)$. 

\paragraph{Optimise Pseudorange}
The pseudorange between each pair of receivers along each normal vector $\hat{\eta_s}$ creates an overdetermined linear system that is solved via least squares.
\begin{figure}
\centering
\caption{text}
\label{key}
\includegraphics[width=0.7\linewidth]{ChapterPerception/Figures/solve_distances.jpg}
\end{figure}
\begin{eqnarray}
\Phi &=& \begin{bmatrix}
0 & -1 & 0 & ...\\
0 & 0 & -1 & ...\\
... & ... & ... & ... \\
1 & -1 & 0 & ...\\
1 & 0 & -1 & ...\\
\hdotsfor{4} \\
0 & 1 & -1 & ...
\end{bmatrix} \\
%
\Omega_s &=& \begin{bmatrix}
\beta_s \\
\gamma_s\\
\delta_s \\
\vdots
\end{bmatrix} \\
%
\rho_s &=& \begin{bmatrix}
\rho_{\alpha\omega_1}\\
\rho_{\alpha\omega_2}\\
\hdotsfor{1}\\
\rho_{\omega_1\omega_2}\\
\rho_{\omega_1\omega_3}\\
\vdots
\end{bmatrix} 
\end{eqnarray}

\begin{eqnarray}
\Phi\times\Omega_s = \rho_s
\end{eqnarray}
Solve by linear least squares for an overdetermined system by the pseudo inverse matrix
\begin{eqnarray}
\Omega_s = (\Phi^T\Phi)^{-1}\Phi^T\rho_s
\end{eqnarray}

\subsubsection{Point Optimisation}
\paragraph{Create Planes}
Create sets of planes for each receiver $\omega$ from the normal vectors $\hat{\eta_s}$ and the set of distances from the reference point $\alpha$ to receiver $\omega$ along each of the normal vectors denoted $\Omega_\omega$.

The equation of a plane is $Ax+By+Cz+D=0$ where the coefficients [A,B,C] describe the normal vector of the plane and the coefficient D sets the plane in 3D space along the vector. As the normal vector is already calculated for each satellite, only the D coefficient must be solved for each receiver and satellite pair. 
\begin{eqnarray}
P_\omega^s &=& (i\cdot\hat{\eta_s})x + (j\cdot\hat{\eta_s})y + (k\cdot\hat{\eta_s})z + D_\omega^s \label{genplane}\\
P_\omega^s &=& I\cdot H +D_\omega\\
\end{eqnarray}
Where $I = x\hat{\textbf{i}}+y\hat{\textbf{j}}+z\hat{\textbf{k}}$ is the *identity* vector and H is a matrix of normal vectors to each satellite:
\begin{eqnarray}
H = \begin{bmatrix}
\hat{\eta_1} \\
\hat{\eta_2} \\
\vdots\\
\hat{\eta_n}
\end{bmatrix}
\end{eqnarray}
The coefficient D can be calculated by finding a point on the plane $f_\omega^s$, then substituting it into \eqref{genplane} for x,y,z. The point of the plane is calculated by moving along the normal vector by the optimised pseudo distance from the reference point \eqref{Eq:f}.
\begin{eqnarray}
f_\omega^s &=& \Delta_\omega^s\hat{\eta_s} \label{Eq:f}\\
P_\omega^s &=& \hat{\eta_s}\cdot f_\omega^s +D_\omega^s = 0\\
D_\omega^s &=& -\hat{\eta_s}\cdot f_\omega^s\\
D_\omega^s &=& -\Delta_\omega^s ||\hat{\eta_s}|| \\
||\hat{\eta_s}|| &=& 1\\
D_\omega^s &=& -\Delta_\omega^s\\
\Rightarrow P_\omega^s &=& I\cdot H -\Omega_\omega
\end{eqnarray}

\begin{figure}[h]
\centering
\caption{Find position $f_\omega^s$ on the plane}
\label{fig:pointonplane}
\includegraphics[width=0.7\linewidth]{ChapterPerception/Figures/pointonplane}
\end{figure}






$\Omega_\omega$ is a vector of optimised pseudo-distances from reference $\alpha$ to receiver $\omega$ for all satellites $s\in{1,2...n}$
\begin{eqnarray}
\Omega_\omega = \begin{bmatrix}
\Delta_{\omega}^1 \\
\Delta_{\omega}^2 \\
\vdots\\
\Delta_{\omega}^n \\
\end{bmatrix}
\end{eqnarray}
Where $\Omega_s$ is the vector of optimised pseudo-distances from $\alpha$ to each receiver $\omega\in1,2...m$ for a single satellite s:
\begin{eqnarray}
\Omega_s = \begin{bmatrix}
\Delta_{\omega_1}^s \\
\Delta_{\omega_2}^s \\
\vdots\\
\Delta_{\omega_m}^s \\
\end{bmatrix}
\end{eqnarray}


\paragraph{Solve for Intersection}
As the system of homogeneous linear equations is overdetermined, it can be solved using singular value decomposition to find a point that has the minimum residuals from all of the planes in its set $P_\omega$. Each set of planes for a particular receiver is independent to all other receivers. The vector $X_\omega$ describes the position of receiver $\omega$ in NED coordinates and $\tau_\omega$ describes a final receiver clock bias that alters the displacement of all the planes in the set $P_\omega$ by the same parameter.
\begin{eqnarray}
X_\omega &=& \begin{bmatrix}
x_\omega \\y_\omega \\ z_\omega \\ \tau_\omega
\end{bmatrix}\\
P_\omega X_\omega &=& D_\omega \\
\end{eqnarray}
In order to solve all of the receivers with the least amount of error in the whole system, all of the position vectors $X_\omega$ are solved at the same time. The reference planes of $\alpha$ must be included as a constraint on the system. All of the clock biases are also constrained with the clock bias from $\tau_\alpha$, see \eqref{Eq: P with alphat}.
The receiver clock bias only affects the equation of the planes by altering the constant as a change in the pseudorange has no affect over the angle of the plane. Each receiver clock bias alters all the planes associated with that receiver proportionally. 
\begin{eqnarray}
P_\omega^s &=& (i\cdot\hat{\eta_s})x + (j\cdot\hat{\eta_s})y + (k\cdot\hat{\eta_s})z + D_\omega^s + (\tau_\omega-\tau_\alpha) \label{Eq: P with alphat}\\
\end{eqnarray}
 

