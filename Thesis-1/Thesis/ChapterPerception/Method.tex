%!TEX root = ../Thesis.tex

\section{Tables}\label{sec:exmapletables}

Tables can be pretty messy, and there's a million features in a lot of \LaTeX{} packages, which you'll learn as you need specific things.

The best advice for now is probably to check out the \texttt{booktabs} package, which makes tables look much nicer than the `usual' way of making tables in \LaTeX{}. They have a few guidelines/rules, like ``don't use vertical lines'', and simplify some of the process of putting the horizontal lines in tables.

\begin{table}[H]
    \caption[Example Table]{A simple example of a table to describe something}
    \label{tab:example}
    \begin{center}
        \begin{tabular}{r c p{11em}}
            % columns specified by alignment, p for paragraph with a set width
            \toprule
            % toprule, midrule and bottomrule provide nice looking horizontal rules
            % with appropriate thicknesses and spacings above/below
            \textbf{Label} & \textbf{Value} & \textbf{Description} \\
            \midrule[1pt]
            % i've thickened this midrule as I think it looks nicer, but its still a midrule since
            % it's in between two rows of text (a top/bottom rule would have awkward spacing
            % above/below the line)
            % you don't need to put a midrule between each line, it often looks better without
            Width & 3~cm & The width of the object \\
            Height & 8~cm & The height of the object \\
            Depth & 14~cm & The depth of the object \\
            \midrule
            Shape & ovoid & The geometric shape descriptor of the object \\
            \bottomrule
        \end{tabular}
    \end{center}
\end{table}

In this case, I've forced the table placement using the \texttt{H} placement argument. This is perhaps a bit more common for simple tables where maybe you just want to make a brief point in a slightly more graphically structured way.

In \autoref{tab:example2}, some of the nice features of the \texttt{tabularx} package are demonstrated, which lets you set properties on a column, and simplifies text-wrapping combined with auto-width calculations. This table is also placed using \texttt{htb} rather than \texttt{H}, so is properly floating---note the effect on the line spacing (the rest of the document has a longer line spacing, which doesn't infiltrate floats).

\begin{table}[htb]
    \caption[A More Complex Table Example]{A slightly more complex example table, this time using
        features from the \texttt{tabularx} package.}
    \label{tab:example2}
    \begin{center}
        \begin{tabularx}{\textwidth}{>{\bfseries}l X}
            % note the extra argument for the tabularx environment to specify the table width
            % the X column then just expands out to this width
            \toprule
            \textbf{Bold} \textbf{Text Wrap and Fill Width} \\
            \midrule
            Note & It's much easier to have a column that fills the width of the table and also
            provides text-wrapping using the \texttt{tabularx} package and the \texttt{X} column
            specifier, than it is with the regular \texttt{p} column specifier. \\
            \midrule
            Also Note & It's also easy to make elements in a column automatically bold with
            \texttt{tabularx}, without having to specify them as bold on each line. \\
            \bottomrule
        \end{tabularx}
    \end{center}
\end{table}


